\chapter*{Streszczenie}

TOOO
Analiza istniejących projektów na githubie. sukcesy i porażki
Ile projektów jest **** a ile upadło.
Jak wygląda historia commitów jednych i drugich.

Teza1: najpierw jest super pommysł, nie ważne jak wygląda kod, dokumentacja, strona domowa, dopiero potem, jak ludzie chca kontrybuować, to jest tzw wersja 2.0, gdzie jest dużo tooli i wszystko smiga

Teza2: nawet pomysl moze nie byc super hiper, ale tool jest wiarygodny spelnia standardy, jest dostepny z bowera itd. wiec uzytkownicy chca to lubia.


The \textbf{Streszczenie} should correspond to the \textbf{Abstract} and includes the same key elements. It  should be prepared according to faculty rules. The \textbf{Streszczenie} is a chapter written only by Polish authors who decide to write the diploma thesis in English and it is not applicable to foreign students. 

\vspace{12pt}
\noindent\textbf{Słowa kluczowe}: [keywords] Zażółć gęślą jaźń.

\vspace{12pt}
\noindent\textbf{Dziedzina nauki i techniki, zgodnie z wymogami OECD}: \textless{}dziedzina\textgreater{}, \textless{}technika\textgreater{},\ldots
[field of science and technology in accordance with OECD requirements: \textless{}field\textgreater{}, \textless{}technology\textgreater{},\ldots]