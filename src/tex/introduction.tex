\chapter{Wstęp}

Produktem pracy będzie biblioteka w języku JavaScript do wykorzystania w zaawansowanych aplikacjach webowych w przeglądarce internetowej.


Kilka uwag do tematu pracy magisterskiej:
\begin{itemize}
	\item Pomysł na temat pracy jest związany z moim doświadczeniem zawodowym zdobytym w firmie DAC System, a problemy technologiczne wskazane poniżej istotnie wpływają na efektywność pracy monitoringu jakości powietrza w Wojewódzkich Inspektoratów Ochrony Środowiska, czyli użytkowników produktu, który rozwijałem we tej firmie.
	\item Część poruszonych problemów została przeze mnie rozwiązana w ramach pracy w firmie DAC System, i zgodnie z sugestią firmy włączę je do swojej pracy.
	\item Mowa tutaj o danych pomiarowych, czyli danych zbieranych przez urządzenia pomiarowe oznaczone stemplem czasowym, zbieranym przez wiele lat co kilka sekund. Jest to zagadnienie związane z tzw. Big Data
	\item Produktem końcowym jest arkusz danych pomiarowych edytowany analogicznie do edycji w programie Excel (zdjęcie poniżej) i wykres liniowy pomagający wyłapać błędne pomiary. Oba produkty dostosowane będą do płynnej obsługi wielkich zbiorów danych pomiarowych.
	\item Główne wyzwania do podjęcia:
		\begin{itemize}
			\item Zaprojektowanie protokołu komunikacyjnego klient-serwer opartego na technologii AJAX, którego zadaniem jest ukrycie przed użytkownikiem faktu, że pracuje on tylko na pewnym wycinku danych. Z tego protokołu będą korzystały komponenty graficznego interfejsu użytkownika.
			\item Optymalizacja procesu renderowania komponentu arkusza danych, by wydajnie obsługiwać lokalne zbiory danych niezależnie od ich rozmiaru.
			\item Zaprojektowanie komponentu wykresów liniowych, pozwalającego na płynne przeglądanie danych pomiarowych, minimalizującego czas oczekiwania użytkownika na dane.
		\end{itemize}
	\item Przypadki użycia które postawiły te wyzwania technologiczne:
		\begin{itemize}
			\item Użytkownik chce płynnie przeglądać dane z dowolnego zakresu czasowego
			\item Kiedy użytkownik powraca do danych już odwiedzonych (np. przewijając wykres w przód lub w tył), dane powinny być odtworzone z pamięci lokalnej.
			\item Niemożliwe jest pobranie wszystkich danych do klienta na jego żądanie
			\item Klient powinien dokonywać predykcji, które fragmenty danych będą za chwilę potrzebne, aby minimalizować czas oczekiwania na pobranie następnego kawałka danych
			\item Klient powinien dbać o zwalnianie zasobów z pamięci na podstawie informacji, które fragmenty danych były użyte najdawniej / najrzadziej.
		\end{itemize}
	\item Rozwiązanie protokołu komunikacji podobne jest do tego stosowanego w mechanizmie używanym przez Mapy Google, w kontekście ładowania kafelków mapy.

\end{itemize}

Oprócz wytworzenia narzędzia w JavaScript, chcę skupić się na tym, jak przygotować swoją pracę dla innych, by z niej mogli skorzystać inni użytkownicy serwisu http://github.com/. Chodzi o czytelność i wiarygodność projektu. Mam tutaj na myśli:
\begin{itemize}
	\item wyodrębnienie mniejszych, samodzielnych i użytecznych  modułów, które rozwiązują pojedyncze problemy programistyczne
	\item wersjonowanie zgodnie z konwencją Semantic Versioning 2.0.0 (http://semver.org/)
	\item jasno zdefiniowane i opisane interfejsy programistyczne
	\item korzystanie ze standardów - dostępność modułów w czasie wykonania: AMD (np. http://requirejs.org/) oraz w środowisku wytwórczym: http://bower.io/ lub http://npmjs.com/
	\item dbałość o kod (m.in brak polskich wstawek w komentarzach), staranne oznaczanie zmian w repozytorium GIT
	\item jakość: pokrycie testami jednostkowymi i funkcjonalnymi, ciągła integracja (ang. contignous integration)
	\item dostosowanie się do obecnych trendów programowania w JavaScript: EcmaScript6, BabelJS, Gulp 


\end{itemize}



>>>>

Niniejsza praca będzie poświęcona głównie ideologii Open Source, czyli tej, w której twórcy oprogramowania dzielą się swoim dziełem udostępniając wszystkie zasoby źródłowe, przyczyniając się tym samym do rozpowszechniania wiedzy i praktyk.

Osoby zainteresowane tym dobrem wspólnym skupiają się wokół specjalistycznych portali internetowych tworząc tym samym specyficzną społeczność. Jedną z nich jest społeczność portalu \textbf{github.com}, który jest obecnie powszechnie znanym zbiorem repozytoriów, miejscem, w którym utrzymywane są najważniejsze i najpopularniejsze biblioteki programistyczne.

Korzyści płynące z korzystania z darmowych, utrzymywanych i rozwijanych, wiarygodnych projektów są zgoła oczywiste. Inaczej jest z korzyściami płynącymi z twórczego wkładu w Otwarte Oprogramowanie. Jedni chcą móc się pochwalić autorstwem powszechnego narzędzia, inni chcę mieć wartościowy wpis w życiorysie, robią to w ramach pracy, a jeszcze inni czują po prostu potrzebę podzielenia się swoim dziełem z innymi.
<<< aspekt zopensourcownani
W tej pracy postawię się w roli programisty chcącego utworzyć nowy projekt open source, który będzie współrozwijany przez społeczność GitHuba, zyska na popularności i nabierze własnego rytmu. Niestety, można się domyśleć, że do osiągnięcia opensourcowego sukcesu nie wystarczy załadować swój "idealny" projekt do repozytorium GitHuba. Nawet najlepsze pomysły, którmi chce się podzielić twórczy programista, są skazane na brak zainteresowania, jeśli nie są dostosowane do wymagań innych użytkowników. Aby sprawdzić, czym powinien charakteryzować się nowy, nieznany jeszcze projekt, będę musiał postawić się również w roli programisty, który przypadkiem znalazł ciekawą bibliotekę nadającą się do jego projektu, ale która w ogóle nie jest znana, co wiąże się zwiększonym ryzykiem porażki całego projektu. Tylko wtedy, gdy sam użyje tej biblioteki, może stwierdzić, że warto poświęcić swój czas na jej rozwijanie.

Postaram się znaleźć dobre odpowiedzi na poniższe pytania:
\begin{itemize}
\item kiedy bibliotekę otwartoźródłową społeczność uważa za godną użycia,
\item jakie są zagrożenia płynące z użycia niewiarygodnej biblioteki,
\item czego wymaga się od bibliotek rozwijanych przez innych.
\end{itemize}
Zanim rozpocznę tworzenie projektu open source na portalu github.com, przeprowadzę krótką analizę, by dowiedzieć się, czym charakteryzowały się w swojej początkowej fazie projekty na tym portalu, które obecnie zyskały na popularności. Ponieważ biblioteka, którą chcę tworzyć, będzie napisana w języku JavaScript, w analizie ograniczę się do małych bibliotek również napisanych w tym języku. Dzięki temu, z zebranych spostrzeżeń będę mógł wysnuć wnioski pomocne przy planowaniu projektu.

Na wstępie warto zaznaczyć, że pomysł realizowany w projekcie powinien być innowacyjny, nie powinien być tylko alternatywą istniejącego rozwiązania, gdyż można się domyśleć, że ludzie mając porównanie z popularną biblioteką, będą kierowali się raczej tym ostatnim w kryteriach swojego wyboru.
>>>>odtąd prawdziwy wstęp
Projekt będzie miał na celu wytworzenie narzędzia programistycznego rozwiązującego problemy występujące w bliskiej mi branży telemetrii i monitoringu środowiska. Problemy te poznałem zdobywając doświadczenie w firmie DAC System, gdzie brałem udział w wytwarzaniu systemu CAS14, tzw. \textbf{grubego klienta} systemu monitoringu jakości powietrza dla ośmiu Wojewódzkich Inspektoratów Ochrony Środowiska, zamawianych przez Główny Inspektorat Ochrony Środowiska w zamówieniu \textsl{ZP/DM/0811-07-EOG/01/2013/MZ}.

Część tych problemów rozwiązałem w tamtym systemie, dlatego za zgodą i jednocześnie sugestią ze strony firmy DAC System, przeniosę część rozwiązań do narzędzia otwartego.

%Blabla jak ważne są dane pomiarowe i efektywne ich użycie, potem że skoro chcę rozwiązać problemy, które nie wydają się być jeszcze dobrze rozwiązane, warto się podzielić ze światem, ale to nie jest takie oczywiste.

\section{Cel pracy}

\subsection{Cel biznesowy i Open Source}

\subsubsection*{Cel biznesowy}
%wydajny monitoring i walidacja danych pomiarowych
\subsubsection*{Cel Open Source}
%rozpowszechnienie, interakcja z githubem, kontrybucje

\subsection{Produkt pracy}
%biblioteka programistyczna i przykładowa aplikacja demonstracyjna (użycie biblioteki)
%na jakich licencjach wydano produkty - i dlaczego

\subsection{Charakterystyka Open Source}

%definicja sukcesu projektu otwartoźródłowego

\section{Motywacja}

%Przyczyny podjęcia się tematu

\subsection{Monitoring środowiska}
%implementacja wkleszczona w system CAS14, bardzo pod klienta, a warto takim czymś podzielić się ze światem, bo tego nie ma. implementacja korzysta z bibliotek dostępnych w tamtym systemie, a których licencja nie pozwala na otwarcie kodu źródłowego

\subsection{Stan wyjściowy}
%arkusz danych w ExtJS, ściśle zależny od innych modułów aplikacji CAS14 firmy DAC System, co dokładnie robi kod i co kopiuje do magisterski, a czego nie, co dodatkowo będzie trzeba zrobić
%jak wygląda inicjalnie repozytorium GitHuba

\section{Główne wyzwania}
%zainteresowanie społeczności programistów na portalu GitHub
%kolaboracja - rezerwowanie obszarów na wyłączność, obszary niekonfliktowe
%powiadomienia o zmianach
%pokrycie testami - wiarygodność w open source
%lokalna historia - cofnij i powtórz
%ux arkusza
%wielkość danych
% * efektywne przeglądanie danych historycznych
% * cache i prefetch
% * infinity data jak w google maps
\section{Aspekt badawczy}
%było dużo we wstępie
\section{Wykorzystane narzędzia}
% spring boot, es6, angular, webpack, babel
%spring boot, es6, angular, webpack, babel
%github, może heroku, może jakiś remote build system
%Java/Javascript/compass
\section{Wykorzystane wzorce projektowe}
