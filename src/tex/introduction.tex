\chapter{Wstęp}


Niniejsza praca będzie poświęcona głównie ideologii Open Source, czyli tej, w której twórcy oprogramowania dzielą się swoim dziełem udostępniając wszystkie zasoby źródłowe, przyczyniając się tym samym do rozpowszechniania wiedzy i praktyk.

Osoby zainteresowane tym dobrem wspólnym skupiają się wokół specjalistycznych portali internetowych tworząc tym samym specyficzną społeczność. Jedną z nich jest społeczność portalu \textbf{github.com}, który jest obecnie powszechnie znanym zbiorem repozytoriów, miejscem, w którym utrzymywane są najważniejsze i najpopularniejsze biblioteki programistyczne.

Korzyści płynące z korzystania z darmowych, utrzymywanych i rozwijanych, wiarygodnych projektów są zgoła oczywiste. Inaczej jest z korzyściami płynącymi z twórczego wkładu w Otwarte Oprogramowanie. Jedni chcą móc się pochwalić autorstwem powszechnego narzędzia, inni chcę mieć wartościowy wpis w życiorysie, robią to w ramach pracy, a jeszcze inni czują po prostu potrzebę podzielenia się swoim dziełem z innymi.

W tej pracy postawię się w roli programisty chcącego utworzyć nowy projekt open source, który będzie współrozwijany przez społeczność GitHuba, zyska na popularności i nabierze własnego rytmu. Niestety, można się domyśleć, że do osiągnięcia opensourcowego sukcesu nie wystarczy załadować swój "idealny" projekt do repozytorium GitHuba. Nawet najlepsze pomysły, którmi chce się podzielić twórczy programista, są skazane na brak zainteresowania, jeśli nie są dostosowane do wymagań innych użytkowników. Aby sprawdzić, czym powinien charakteryzować się nowy, nieznany jeszcze projekt, będę musiał postawić się również w roli programisty, który przypadkiem znalazł ciekawą bibliotekę nadającą się do jego projektu, ale która w ogóle nie jest znana, co wiąże się zwiększonym ryzykiem porażki całego projektu. Tylko wtedy, gdy sam użyje tej biblioteki, może stwierdzić, że warto poświęcić swój czas na jej rozwijanie.

Postaram się znaleźć dobre odpowiedzi na poniższe pytania:
\begin{itemize}
\item kiedy bibliotekę otwartoźródłową społeczność uważa za godną użycia,
\item jakie są zagrożenia płynące z użycia niewiarygodnej biblioteki,
\item czego wymaga się od bibliotek rozwijanych przez innych.
\end{itemize}
Zanim rozpocznę tworzenie projektu open source na portalu github.com, przeprowadzę krótką analizę, by dowiedzieć się, czym charakteryzowały się w swojej początkowej fazie projekty na tym portalu, które obecnie zyskały na popularności. Ponieważ biblioteka, którą chcę tworzyć, będzie napisana w języku JavaScript, w analizie ograniczę się do małych bibliotek również napisanych w tym języku. Dzięki temu, z zebranych spostrzeżeń będę mógł wysnuć wnioski pomocne przy planowaniu projektu.

Na wstępie warto zaznaczyć, że pomysł realizowany w projekcie powinien być innowacyjny, nie powinien być tylko alternatywą istniejącego rozwiązania, gdyż można się domyśleć, że ludzie mając porównanie z popularną biblioteką, będą kierowali się raczej tym ostatnim w kryteriach swojego wyboru.

Projekt będzie miał na celu wytworzenie narzędzia programistycznego rozwiązującego problemy występujące w bliskiej mi branży telemetrii i monitoringu środowiska. Problemy te poznałem zdobywając doświadczenie w firmie DAC System, gdzie brałem udział w wytwarzaniu systemu CAS14, tzw. \textbf{grubego klienta} systemu monitoringu jakości powietrza dla ośmiu Wojewódzkich Inspektoratów Ochrony Środowiska, zamawianych przez Główny Inspektorwat Ochrony Środowiska w zamówieniu \textsl{ZP/DM/0811-07-EOG/01/2013/MZ}.

Część tych problemów rozwiązałem w tamtym systemie, dlatego za zgodą i jednocześnie sugestią ze strony firmy DAC System, przeniosę część rozwiązań do narzędzia otwartego.




\section{Cel pracy}

\subsection{Cel biznesowy i Open Source}

\subsubsection*{Cel biznesowy}

\subsubsection*{Cel Open Source}


\subsection{Produkt pracy}

\subsection{Charakterystyka Open Source}

\section{Motywacja}

Przyczyny podjęcia się tematu

\subsection{Monitoring środowiska}

\subsection{Stan wyjściowy}

\section{Główne wyzwania}

\section{Aspekt badawczy}

\section{Wykorzystane narzędzia}

\section{Wykorzystane wzorce projektowe}




The general formatting requirements for the diploma thesis are listed below:
\begin{itemize}
	\item sheet size: A4,
	\item paper orientation: vertical,
	\item font: Arial,
	\item basic font size: 10 pt.,
	\item line spacing: 1.5,
	\item mirror margins:
	\begin{itemize}
		\item top: 2.5 cm,
		\item bottom: 2.5 cm,
		\item internal: 3.5 cm,
		\item external: 2.5 cm,
	\end{itemize}
	\item the thesis text should be justified (aligned to both margins),
	\item each paragraph should begin with a 1.25 cm indentation.
\end{itemize}

The thesis should be prepared for double-sided printing. The page numbering should be in the page footer and centred. The title page should include the author’s (authors’) Statement (Statements) and the page number should not be printed. Page numbering, should be in Arabic numerals using a 9 pt. Arial font. It should begin on page 3 (the Table of Contents) and continue to the last page.

An example of the correct way of presenting information in points (bulleted list) is shown above. Each point (line of text) should be preceded by a bullet. It should begin in the lower case and end with a comma or semi-colon, except for the last point (line of text), which should end with a full stop.

The title of a table should be directly above the table, with a 9 pt. font size and not ended with a full stop, as shown above. Paragraph spacing for the text in the table is as follows:
\begin{itemize}
	\item top 6 pt.,
	\item bottom 0 pt.
\end{itemize}

The correct headings are presented in Table~\ref{tab:heading-styles}.

\begin{table}[h]
	\caption{Sizes and styles of headings}
	\label{tab:heading-styles}
	\begin{tabularx}{\textwidth}{|X|X|X|}
		\hline
		Level of heading	& Example 					& Font size and style \\ \hline
		Heading 1 			& \textbf{1. CHAPTER TITLE}			& 12 pt., CAPITALS, bold \\ \hline
		Heading 2			& \textbf{\textit{1.1. Subchapter title}}		& 10 pt., bold and in italics \\ \hline
		Heading 3			& \textit{1.1.1. Subchapter section}	& 10 pt., italics \\ \hline
	\end{tabularx}
\end{table}

Data should be presented in the table as in Table~\ref{tab:heading-styles}, shown above, i.e. using a 9 pt. font and aligning text to the left edge of the cell.

Table numbering is continuous within the chapter. The table sequence number (table title) is preceded by the word Table and the number of the chapter, ended with a dot (e.g. Table~1.1. Size\ldots). Every table must be referred to in the thesis text, e.g.`Table~\ref{tab:heading-styles}. contains\ldots'.

If a table needs to be continued on more than one page, the table heading should appear on each subsequent page, using the option: TABLE PROPERTIES -> row -> repeat as heading row at the top of every page.

The first paragraph below a table should begin with a top margin of 12 pt.

Lines of text should not end with short prepositions, such as: a, an, the, in, on, etc. In such cases the non-breaking space (NBSP), using ctrl, shift and space, is recommended instead of an ordinary space.
