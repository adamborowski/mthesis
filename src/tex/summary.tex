\chapter{Summary}

%      - realizacja założonych celów
%      - odejścia od założeń
%      - operational transformation nie jest potrzebne, raczej rozwiązywanie konflikty offline
%      - problemy napotkane podczas realizacji pracy 
%      - możliwe kierunki rozwoju
%      - wnioski


%TODO przeczytać, sformatować i rozdzielić do rozdziałów, a to co sie powtarza - usunąć!
%Big data? Big measurement data? Jest coś o gym w Internecie?
%
%Trochę o UX i jak to jest z nawykami ludzi? Nawiązanie do Excela
%
%Dane historyczne? Po co? W jaki sposób z nich skorzystamy i jak to zoptymalizować
%
%We wstępie można zrobić disclaimer, że nie warto robić operational transformation, bo klient tego typu tego raczej nie potrzebuje.
%
%
%Co ma zostać wytworzone?
%
%Serwer aplikacji Spring Boot - RESTy
%Moduł danych - implementacja interfejsu GitDrivenDataStore extends DataStore
%Generator danych podłączony jako źródło danych - osobna aplikacja?
%Komponent JS/CSS konfigurowalny JSONem do wykorzystania w aplikacjach - właściwy przedmiot magisterki
%subkomponent DataStore jak dla google Maps - jedno i dwu wymiarowy
%Portal WWW klienta, umożliwiający wybór danych do przeglądania może github.io?? + heroku
%serie pomiarowe
%zakres czasowy
%Co ważne w kontekście githuba i open source:
%Ważne, aby publikować jak najbardziej niezależne produkty, które rozwiązują jasno konkretne produkty, aby użytkownicy nie bali się żadnej czarnej magii pod spodem.
%Ważne, aby były co najmniej unit testy
%Nasze narzędzie ma być integracją abstrakcyjnych interfejsów i potem możemy podpinać implementacje. Tak samo ma być z serwerem - implementacja git
%ważne, aby była dokumentacja API
%github.io - stronka demo
%
%
%Co warto sprawdzić?
%ExtJS - like config system for ES6 - może zaimplementować?
%
%
%Co można draftować
%Dokumentację implementacji serwisu danych (w gicie)
%
%Analiza obecnych rozwiązań - pisałem przecież z kimś meile do big measurement data	
%
%zrobić draft jak ma wygladac repozytorium, do kazdego modulu testy, wersje snapshot jak w mvn, jedno repozytorium, ale wiele artefaktow, osobno budowane, i glowna apka wszystko sobie zaciaga…
%jak to zrobic by testy byly dobrze zintegrowane z IDEA
%
%https://developers.google.com/speed/articles/prefetching?hl=en
%
%https://scholar.google.pl/scholar?q=time+based+data+prefetching+javascript&hl=en&as_sdt=0&as_vis=1&oi=scholart&sa=X&ved=0ahUKEwjV0qvun6nKAhXrv3IKHeX-DT0QgQMIHDAA
%
%contignous data prefetching, caching, etc
%
%
%
%dlacczego warto?
%napisać, że są branże które takich rzeczy potrzebuję
%napisać, że deweloperzy mogliby skorzystać z tych dobrodziejstw, bo nie ma takich rzeczy
%
%Jakie rozdziały?
%
%Wstęp
%Cele pracy - „Stowrzenie" i „Rozpowszechnienie”
%Stworzenie
%Produkt w ujęcu biznesowym - jak będzie można wykorzystać to zostało stworzone
%Przypadki użycia - myślimy o klientach końcowych
%Produkt w ujęciu programowym - komponenty JavaScript (ContignousDataStore, Chart, Sheet) - fajne moduły - myślimy o programistach
%Demonstracja - integracja wszystkich komponentów w jeden, który rozwiąże problem biznesowy - WIOŚ
%Rozpowszechnienie
%Specyfika open source
%Aspekt psychologiczny: nie chcę ryzykować polegając na niewiarygodnych bibliotekach
%Biblioteka musi rozwiązywać pojedynczy cel, na przykład: contignous data prefetching and caching
%Musi być pokryta testami - wiarygodność
%Moduły Programowe
%ContignousDataStore
%jak googlel maps
%EditableContignousDataStore
%kolejna warstwa, która pod spodem używa ContignousDataStore (ale nie wymaga, tj zrobić na zasadzie adaptera, który pośredniczy)
%tutaj musimy uwzgędnić, że dane załadowane z serwera mogą zostać zmienione przez użytkownika
%co to znaczy? że np muszą być eventy które powiadomią, że coś zostało zmienione.
%ta warstwa przechowuje elementy edytowane, a jak element nie był edytowany, to rób fallback do datastore.
%tutaj też trzeba trzymać stan obiektu: changed/progress/[jak się zsynchrronizuje to usuwamy]
%jakoś trzeba tutaj pomyśleć, jak DataStore ma przeładować niektóre elementy? niech do DataStore podajemy DataProvider (czyli implementacja skąd ajaxujemy ) i zwracamy kolekcję nowych obiektów — DataStore powinien obsługiwać, że serwer zwraca kawałki danych. Wtedy robimy dataStore.request() - a my już wiemy jak odpowiedzieć.
%
%


Every diploma thesis must include a chapter entitled \textbf{Summary}. It should appear before the \textbf{Bibliography} and include a review of the main points of the thesis and/or obtained results. The chapter should also state what should be realized if research into the subject of the thesis is continued.